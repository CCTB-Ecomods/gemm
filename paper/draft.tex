\documentclass[a4paper]{scrartcl}
\usepackage[english]{babel}
\usepackage[
  backend=biber,
  style=authoryear
]{biblatex}
\addbibresource{refs.bib}

\usepackage{graphicx}
\usepackage[T1]{fontenc}
\usepackage[utf8]{inputenc}
\usepackage{csquotes}
\usepackage{cleveref}

\title{The relative roles of genetic, ecological and environmental factors in insular species radiations}
\author{Ludwig Leidinger}
\date{\today}
%

\begin{document}
\maketitle

\begin{abstract}
  Islands are known for their peculiar and fascinating biotas, typically very distinct from any other landmasses, be it neighboring islands or continental areas.
  Although island biogeography theory has seen an ever increasing development over the years,
  the mechanisms underlying these patterns are not fully understood/ not yet explained in a direct causal relationship.
  Specifically, some processes remain un- or under-explored, foremost of these adaption and micro-evolution.
  We use an individual-based mechanistic model implementing genetic and micro-evolutionary processes and other commonly investigated ecological and
  environmental processes to assess the importance of drivers across different organisational and spatio-temporal levels
  on species trait distribution on islands.
  Specifically, we address the question why some lineages diverge stronger than others.
  We found that lineages with intermediate levels of gene linkage evolve into a higher number of distinct populations
  than low or high gene linkage lineages.
  The model enabled us to also take a closer look on the properties of lineages/species that actually managed to colonize islands,
  and which of them resulted in radiations.
\end{abstract}

\section{Introduction}
%% Intro subsections (==paragraphs):
%% 1. Major topic (include only background relevant for exp. design)
%% 2. Islands
%% 3. Forces
%% 4. Models
%% 5. Aim, Objectives
%% 2+3 may be combined, as well as 1+2(+3)

%% Cite & look at Katja Schiffer paper
%% "species rich vs. species poor lineage''
%% ``emerging systematic, functional, phylogenetic patterns''
%% Hurlbert

%% major topic: besides classical mechanisms (ecological, environmental), genetic mechanisms affect speciation/radiations
Island radiations like the Darwin finches on Galápagos or \textit{Aeonium} on the Canaries have fascinated researchers throughout history %% cite? Darwin? ...?
To this date, however, it is unclear why these genera produced this multitude of species as opposed to most other species groups.
%%
Nevertheless, an increasing number of studies investigates genetic processes and
their effect on reproductive/ecological isolation of populations (e.g., Via \& West 2008, Via 2001 and references therein)
as well as evolutionary rescue provided by the feedback with ecological processes (Schiffers et al. 2013).
%%TODO: here come a list of processes with examples
%% - genetic processes in general
%% - isolation processes
Besides arthropods (Howthorne \& Via 2001), these processes can be observed in plants as well. %Are you talking about islands here? Be clearer, because you said about the lack of geentic studies in island contexts.
For example, the rapid switching of pollinator syndromes in species of the genus \textit{Petunia} %use package for this!
seems to be a consequence of the tight genetic linkage between loci for pollination syndrome traits (Hermann et al.\ 2013). % cite properly!
Considering these phenomena, we expect speciation rates to be highest at intermediate levels of gene linkage --- low
levels leads to dissolving phenotype populations, whereas high levels hinder rapid adaption. %what Schiffers et al say about linkage and evolutionary rescue (basically the opposite of diversification?)

%% Islands
Islands are understood as suitable model systems for ecology and evolution due to their well definable characteristics
like isolation (e.g. Losos \& Ricklefs 2009; Warren et al. 2015  %%Islands as model systems in ecology and evolution: prospects fifty years after MacArthur‐Wilson. EcoLett). %TODO: ref
Part of this value as study system originates by typically remarkable species.
In fact, islands often possess unique species communities and can exceede mainland areas
by a factor of 9.5 and 8.1 in plant and animal  endemism, respectively
(Kier et al. 2005. %A global assessment of endemism and species richness across island and mainland regions. PNAS). %TODO: refs
Such high endemism makes islands  especially threatened by human impacts, in particular climate change and invasive species (Fordham \& Brook 2010). %TODO: refs
Hence, understanding the origin, maintenance, functioning and evolution of island communities is of a high priority objective.
For the prediction of island community size, several theoretical frameworks have been proposed,
including one of the first and to this day still highly influential classical ecological theories,
namely MacArthur and Wilson's equilibrium theory of island biogeography (ETIB). %TODO: ref
In it, species richness on islands is the result of opposing rates of immigration and extinction,
both of which in turn are modified by island size (area) and its distance from the mainland (isolation).
This relationship was since successfully applied to a multitude of islands (Lomolina \& Weiser 2001, Triantis et al.\ 2011)%TODO: refs
, but also island like systems, such as coral reefs, lakes/ponds or forest fragments (Kelly et al.\ 1989, Peay et al.\ 2007, ). %TODO: refs
This general applicability (Lomolino 2000), however, cannot hide the fact
that the ETIB fails to formalize a mechanistic explanation of how extinction and immigration are modified by area and isolation, respectively.
Additionally, it has been demonstrated more recently that area also affects immigration via target effects (REFs) a
nd isolation affects extinction via rescue-effects (REFs).
Finally, ETIB has particularly been criticized for its lack of considering speciation (see e.g.\ Losos \& Schluter 2000). %TODO: refs
About a decade ago, island biogeography theory gained new momentum by Whittaker et al.'s General Dynamic Model (GDM, Whittaker et al.\ 2007, 2008). %TODO: ref
The conceptual framework traces island communities along the geomorphological trajectory of hotspot islands,
whith islands emerging from sea level via volcanic hotspot activity,
increasing over time in size and topographic complexity,
then eroding and subsiding in the ocean again when the island is not over the hotspot anymore.
In doing so, the model incorporates concepts of niche theory, speciation and species succession.
Yet, there are still many processes and drivers deemed important,
which are not considered by any formal theory and often little investigated
%[if you say this, you have to mention at leastr a couple os examples, otherwise it is an empty claim].

%% Models
The problem of investigating these processes is the time frame and study extent needed for a proper evaluation.
Because islands biogeography takes place at evolutionary time frames (millions of years),
comprehensive and meaningful data is difficult and virtually impossible to obtain.
Therefore, there has been an interest in building predictive models for islands.
In a recent review, we indeed found a rapid increase in number of mechanistic simulaiton models for island biogeography %\cite{leidinger2017biodiversity}
(Leidinger \& Sarmento Cabral 2017).
Although many of the reviewed models integrate multiple processes at a time, unfortunately,
most models failed to include microevolutionary or more explicit genetic processes %\cite{leidinger2017biodiversity}
(Leidinger \& Sarmento Cabral 2017).
One direction towards this is given by models using concepts such as ``protracted speciaton'' (Rosindell \& Harmon 2011) %\cite{rosindellXXX},
where speciation occurs with a delay from the actual
divergence event if the population remains isolated during that time.
But nature is usually more complex than this (cf.\ Silvertown 2004, but see Herben et al.\ 2005)
and genetic mechanisms should also play a central role for speciation/radiation and species richness.
Questions, such as "[cite two questions from Patino et al. 2017]", thus remain largely open (Patiño et al.\ 2017)
and may require explicit consideration of evolutionary processes.
%% Yet, there are numerous examples for fascinating patterns of trait distributions and evolution on islands, such as island syndromes[, ...] or the fact of explosively radiating lineages (Hughes \& Eastwood 2006, Kocher 2004)
%%, while others remain one single species (Price/Chase) (eg. \cite{price2004speciation}).
%% Also: Ecological Opportunity and Adaptive Radiation, Stroud \& Losos 2016: ``Does the concept have predictive value[...]?'' \cite{stroud2016ecological}
%% Phylogenetic niche conservatism, phylogenetic signal and the relationship between phylogenetic relatedness and ecological similarity among species, Losos 2008: ecological vs. phylogenetic similarity \cite{losos2008phylogenetic}
Still, the most mechanisms integrated in island models [?] so far include only ecological
or geo-morphological processes (Rundell \& Price 2009, Esselstyn et al. 2011),
while genetic properties are hardly recognized in the island context.
This is surprising given the essential role islands have played in the development of evolutionary theories.

%Another shortfall of the reviewed models was the lack of integrating ecological niches and adaption.

%Besides adaptive radiations(/evolution), changes in or different measures of geographic isolation (landscape complexity) can also give rise to non-adaptive radiations.

%% Aim, objectives:
%% In this study, we aim to address...
To address this hypothesis, 
we present a mechanistic model that overcomes above-mentioned shortcomings (LIST THEM HERE IN SHORT)
by implementing genetic processes on a niche-explicit background
using an island biogeography simulation arena.%% Using an individual-based mechanistic model we investigated how genetic, ecological and environmental factors colonisation and radiation histories on islands,
%% and what trait combinations constitute successful colonizer and radiating species, respectively.
We will address the questions of what relative impact does (a) genetic linkage and
(b) reproductive isolation have on emerging species richenss and composition,
and (c) which genetic and functional trait syndromes %% Which?
distinguish successful colonizer,
strong competitor and diversifying lineages.


\section{Material and Methods}
%% Include simplified model description ``based on ODD''
%% scheduling (order of processes)
%% justify each decision with reference
%% Experimental design
We use an individual-based metacommunity model written in julia %\cite{JULIALANG}.
Each individual is characterized by a set of genetically controlled (i.e. genomically-coded) traits,
which influence ecological processes of
establishment, competition, growth, (density independent) mortality, reproduction and dispersal.
The simulation arena consists of a larger landmass (the mainland) holding an initial random species pool and a smaller,
initially unpopulated island some distance apart (user-defined in the experimental design).
% it would be nice to hav a table with all model configurations that can be defined by the user upon designing the experiment.
% This might help readers see that the model is general enough to be applied to any ecological arena (mainland-island being one).
During the course of the simulation, individuals may disperse from the mainland to the island, thus colonizing it. 
The complete state of the simulation is recorded every 1000 time steps for analysis.

Building on the model of Cabral et al. (2017a, b), most of the biological rates follow the metabolic theory
of ecology (MTE, Brown et al.\ 2004). % instead of saying 'most', say which follow and which don't
In MTE, a biological rate $b$ depends on the temperature $T$ and individual mass $M$, scaling a base rate $b_0$ as:
\begin{equation}
  b=b_0*M^{3\over4}*e^{-{E_A}\over{k_B*T}}
  \label{eq:MTE}
\end{equation}
with $E_A$ the activation energy and $k_B$ the Boltzman constant.

A detailed protocol can be found in the appendix (\cite{grimm2010odd}, XXX).

\subsection{scenarios}
In the simulations, we modify the value and extent of
(a) gene linkage,
(b) reproductive isolation (or reproduction compatibility),
(c) habitat diversity, and
(d) topographical complexity
to asses their relative contribution on community composition.
\cref{tab:scenarios} summarizes the parameters which were varied for the scenarios, their meaning and their values. %TODO: make table.
Each parameter combination was replicated %TODO: how much?
times, setting a different random seed for each replicate.

\subsection{Data recording}
We record the complete state of our simulation world at the start and end of a simulation run and each time the arena changes (geomorphological dynamics).
Additionally, we store the state of the simulation arena each 1000 %TODO: how much exactly?
time steps and record every individual that colonizes the island to track changes.

Using the thus simulated data, we calculate phylogenies of the colonizing lineages, species numbers, trait spaces etc. [WHAT ELSE?]

\section{Results}

\begin{figure}
  %\includegraphics{}
  \caption{Number of linkage units and number of species per lineage XXX timesteps after colonization.
    [Histogram (with threshold category: strongly diversified lineages) or scatter plot. Two panels/color codes: high and low degree of environmental heterogeneity]}
  \label{specieslinkage}
\end{figure}

\begin{figure}
  %\includegraphics{}
  \caption{Trait space characteristics of lineages.
    (A) [Ordination] Trait space of successful colonizers compared to the mainland species pool. Each point represents a lineage.
    Arrows denote the traits.
    (B) [Ordination?] Linkage of traits of successful colonizers(/radiators?). Traits perpendicular to each other are inherited independently,
  same direction indicate tight linkage of traits.}
  \label{traitspace}
\end{figure}

\begin{figure}
  %\includegraphics{}
  \caption{Exemplary phylogenies of lineages without co-occurring competing lineages calculated from neutral sequences of
    extant individuals.
    (A) Lineage with low gene linkage,
    (B) lineage with intermediate gene linkage,
    (C) lineage with high gene linkage.}
  \label{phylogenies}
\end{figure}

\section{Discussion}
Shortcomings. (/Processes/Factors not considered.)
(Dynamic) resource availability: assumed constant in model, but petentially more variation, complexity in community dynamics.
Interaction: in present model only resource competition, % I would call just say resource competition
but predation, parasitatism, herbivory, mutualism will increase complexity,
and should be rather aimed at future modle developments once mechanistic consequences of competition have been better investigated.
In particular, future model developments may concetrate on pollination,
which may have a high impact on evolution and establishment of both plant and animal species due to specialization. %examples for islands?
%[maybe some words on trophic interactions.]

In reality linkage between genes is not a binary decision, but rather a consequence of the distance between those genes.
The larger the distance, the higher the probability of crossing over during meiosis.
The size of the resulting genomic blocks or islands, however, may increase to several megabases in diverging population ---
a phenomenom coined as ``divergence hitchhiking'' (Via \& West 2008). % cite properly!
This mechanism is thought to facilitate sympatric speciation by reducing the detrimental effects of gene flow and
recombination between populations to speciation (Via 2011, but see Feder \& Nosil 2010).
%During incipient speciation, however, th

In this work we concentrated on genetic influences on speciation.
However, there is ongoing debate on how strongly genetic factors also impact extinction processes (Jamieson 2007).
For example, [discuss the evolutionary rescue studied by Schiffers et al. 2013]
Seeing that observed speciation rates are in fact net speciation rates highlights the need to mechanistically explore
these processes as well.
In fact, genetic processes might play an important role in the issue of extinction debt. %% cite Ludmilla, ask for refs

\printbibliography

\end{document}
