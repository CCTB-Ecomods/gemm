\documentclass[a4paper]{scrartcl}
\usepackage[english]{babel}
\usepackage[
  backend=biber,
  style=authoryear
]{biblatex}
\addbibresource{refs.bib}
\usepackage[T1]{fontenc}
\usepackage[utf8]{inputenc}

\title{The relative roles of genetic, ecological and environmental factors in species radiations}
\author{Ludwig Leidinger}
\date{\today}
%
\begin{document}
\maketitle

\begin{abstract}
  Islands feature peculiar and fascinating biotas, often very distinct from any other island or continental area.
  [Examples of radiations/good and poor radiators, island syndromes etc.]
  Although island biogeography theory has seen an ever increasing development over the years,
  the mechanisms underlying these patterns are not fully understood/ not yet explained in a direct causal relationship.
  Specifically, some processes remain un- or under-explored, foremost of these adaption and micro-evolution.
  We used an individual-based mechanistic model implementing these processes and other commonly investigated ecological and
  environmental processes to assess the importance of drivers across different organisational and spatio-temporal levels
  on species trait distribution on islands.
  This enabled us to also take a closer look on the properties of lineages/species that actually managed to colonize islands,
  and which of them resulted in radiations[~].
  
  
\end{abstract}

\section{Introduction}
island biota have always fascinated and inspired researchers (Darwin, Wallace, ...)

This interest in islands and their recognition as highly suitable model systems resulted in the formulation of
important theoretical frameworks, foremost MacArthur and Wilson's seminal equilibrium theory of island biogeography (ETIB).
Later ... GDM.

These theories have in common a focus on species numbers and respective rates of migration, extinction and speciation.
While they are successful in describing species numbers, they are neutral in design (in the UNTB sense) and thus were not meant to consider species traits and compositions.

Yet, there are numerous examples for fascinating patterns of trait distributions and evolution on islands, such as island syndromes[, ...] or the fact that some lineages radiate explosively following
island colonisation, while others remain one single species (Price/Chase) (eg. \cite{price2004speciation}).

Also: Ecological Opportunity and Adaptive Radiation, Stroud \& Losos 2016: ``Does the concept have predictive value[...]?'' \cite{stroud2016ecological}
Phylogenetic niche conservatism, phylogenetic signal and the relationship between phylogenetic relatedness and ecological similarity among species, Losos 2008: ecological vs. phylogenetic similarity \cite{losos2008phylogenetic}

These patterns have been explained theoretically [here, here and here], but usually one at a time.
In the following, we aim to produce and thus explain these forementioned patterns using a single framework, and thus following the same set of (rather simple) rules.

Using an individual-based mechanistic model we investigated how genetic, ecological and environmental factors colonisation and radiation histories on islands,
and what trait combinations constitute successful colonizer and radiator [this sounds like heating ;)] species, respectively.
\section{Material and Methods}

\section{Results}

\section{Discussion}


\printbibliography

\end{document}
