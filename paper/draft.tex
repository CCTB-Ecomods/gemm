\documentclass[a4paper]{scrartcl}
\usepackage[english]{babel}
\usepackage[
  backend=biber,
  style=authoryear
]{biblatex}
\addbibresource{refs.bib}

\usepackage{graphicx}
\usepackage[T1]{fontenc}
\usepackage[utf8]{inputenc}
\usepackage{csquotes}
\usepackage{cleveref}

\title{The effect of genetic linkage on island endemics richness}%The relative roles of genetic, ecological and environmental factors in insular species radiations}
\author{Ludwig Leidinger}
\date{\today}
%

\begin{document}
\maketitle

\begin{abstract}
  Islands often witness fascinating species radiations, which are usually correlated to the islands' unique
  geo-morphological histories.
  Yet, why these radiations occur in the first place is still under debate.
  Besides eco-environmental factors, there is evidence that genetic characteristics can have important
  effects on the speciation process.
  One of those characteristics is genetic linkage and describes how interdependent genes are known for their peculiar and fascinating biotas, typically very distinct from any other landmasses, be it neighboring islands or continental areas.
  Although island biogeography theory has seen an ever increasing development over the years,
  the mechanisms underlying these patterns are not fully understood/ not yet explained in a direct causal relationship.
  Specifically, some processes remain un- or under-explored, foremost of these adaption and micro-evolution.
  We use an individual-based mechanistic model implementing genetic and micro-evolutionary processes and other commonly investigated ecological and
  environmental processes to assess the importance of drivers across different organisational and spatio-temporal levels
  on species trait distribution on islands.
  Specifically, we address the question why some lineages diverge stronger than others.
  We found that lineages with intermediate levels of gene linkage evolve into a higher number of distinct populations
  than low or high gene linkage lineages.
  The model enabled us to also take a closer look on the properties of lineages/species that actually managed to colonize islands,
  and which of them resulted in radiations.
\end{abstract}

\section{Introduction}

%% Q24. What is the relative importance of ecological versus geographical speciation on islands?
%% => Patino hardly considering genetic mechanisms
%% ``emerging systematic, functional, phylogenetic patterns''
%% Hurlbert

% Rationale: If genetic factors have impact on systems with clearly defined isolation (islands),
% they should be even more important on continental systems.

%% major topic: besides classical mechanisms (ecological, environmental), genetic mechanisms affect speciation/radiations
To this day, island radiations like the classical Darwin finches or Galápagos tortoises %%or \textit{Aeonium} on the Canaries
still engage researchers throughout the world \cite{roman-palaciosTortoiseFinchTesting}.
But why were these lineages able to produce this multitude of species as opposed to most other lineages that remain species poor?
For a population to produce separate species, that population must
(a) have sufficient variability in any one characteristic, e.g. traits or genome, and
(b) have gene-flow restricting factors (e.g. environmental/geographical, ecological or behavioral) that facilitate establishment of distinct distribution patterns of said characteristic.
In islands, as well as continental and aquatic systems, researchers still mainly focus on reproductive barriers like environmental and ecological factors to explain speed and occurences of radiations \cite{krugSpeciesGenusRatios2008}\cite{hurlbertProcessesGeneratingLatitudinal2014}\cite{willisSpeciesDiversityScale2002}. % when radiations DID happen!
In fact, these factors indeed often correlate strongly with the timing of genetic divergence of species and populations \cite{hembryEvolutionaryBiogeographyTerrestrial}. % REFS!
But even then, in some cases the reason for maintenance of population diversity remains in the dark. \cite{chenConcordanceGeneticDiversity}
%% HERE COME SOME EXAMPLES
Additionally, these studies highlight the cases of successful radiations, while a conclusive explanation for why these population variablities and thus radiations could occur in the first place is still missing.
This becomes especially apparent on islands, where species generally experience nearly identical environmental conditions.%REF
Yet, only a fraction of them manage to radiate into multiple species through the course of their evolutionary history \cite{hembryEvolutionaryBiogeographyTerrestrial}.
%mechanisms often fail to explain how the rapid establishment of reproductive barriers or isolation arises,
%which ultimately results in sympatric speciation. %CLAIM! PROVIDE REFERENCES!
%% somewhere here Patino et al 2018 anagenesis, cladogenesis, etc islands
%% Islands:
%% ========
%% Islands are understood as suitable model systems for ecology and evolution due to their well definable characteristics
%% like isolation (e.g. \cite{lososAdaptationDiversificationIslands2009b}; \cite{warrenIslandsModelSystems}.
%% Part of this value as study system originates by typically remarkable species.
%% In fact, islands often possess unique species communities and can exceede mainland areas
%% by a factor of 9.5 and 8.1 in plant and animal  endemism, respectively \cite{kierGlobalAssessmentEndemism2009}
%% Such high endemism makes islands  especially threatened by human impacts, in particular climate change and invasive species \cite{fordhamWhyTropicalIsland2010}
%% Hence, understanding the origin, maintenance, functioning and evolution of island communities is of a high priority objective.
%% [Move this maybe to model section:]
%% For the prediction of island community size, several theoretical frameworks have been proposed,
%% including one of the first and to this day still highly influential classical ecological theories,
%% namely MacArthur and Wilson's equilibrium theory of island biogeography (ETIB). %TODO: ref
%% In it, species richness on islands is the result of opposing rates of immigration and extinction,
%% both of which in turn are modified by island size (area) and its distance from the mainland (isolation).
%% This relationship was since successfully applied to a multitude of islands (Lomolina \& Weiser 2001, Triantis et al.\ 2011)%TODO: refs
%% , but also island like systems, such as coral reefs, lakes/ponds or forest fragments (Kelly et al.\ 1989, Peay et al.\ 2007, ). %TODO: refs
%% This general applicability (Lomolino 2000), however, cannot hide the fact
%% that the ETIB fails to formalize a mechanistic explanation of how extinction and immigration are modified by area and isolation, respectively.
%% Additionally, it has been demonstrated more recently that area also affects immigration via target effects (REFs) a
%% nd isolation affects extinction via rescue-effects (REFs).
%% Finally, ETIB has particularly been criticized for its lack of considering speciation (see e.g.\ Losos \& Schluter 2000). %TODO: refs
%% About a decade ago, island biogeography theory gained new momentum by Whittaker et al.'s General Dynamic Model (GDM, Whittaker et al.\ 2007, 2008). %TODO: ref
%% The conceptual framework traces island communities along the geomorphological trajectory of hotspot islands,
%% whith islands emerging from sea level via volcanic hotspot activity,
%% increasing over time in size and topographic complexity,
%% then eroding and subsiding in the ocean again when the island is not over the hotspot anymore.
%% In doing so, the model incorporates concepts of niche theory, speciation and species succession.
%% Yet, there are still many processes and drivers deemed important,
%% which are not considered by any formal theory and often little investigated
%% %[if you say this, you have to mention at leastr a couple os examples, otherwise it is an empty claim].
A promising avenue to this with growing importance is genetic data and analyses.
Genetic analyses have already helped clearing up phylogenies and island biogeography of many species groups \cite{austinReconstructingIslandRadiation2004}\cite{emersonDiversificationForestBeetle2005}\cite{valenteUsingMolecularPhylogenies2018}%, like bats or cichlids REFS!
and more and more complete genomes are published on a regular basis \cite{alonso-blanco135GenomesReveal2016} \cite{lewinEarthBioGenomeProject2018}.
Some genetic studies from continental species highlight the importance of genetic linkage or ``hitchhiking'' on reproductive/ecological isolation of populations \cite{hawthorneGeneticLinkageEcological2001}\cite{viaGeneticMosaicSuggests2008}
as well as evolutionary rescue provided by the feedback with ecological processes \cite{schiffersLimitedEvolutionaryRescue2012}.
%%TODO: here come a list of processes with examples
%% - genetic processes in general
%% - isolation processes
Besides arthropods \cite{hawthorneGeneticLinkageEcological2001}, these processes can be observed in plants as well. %Are you talking about islands here? Be clearer, because you said about the lack of geentic studies in island contexts.
For example, the rapid switching of pollinator syndromes in species of the genus \textit{Petunia} %use package for this!
seems to be a consequence of the tight genetic linkage between loci for pollination syndrome traits \cite{hermannTightGeneticLinkage2013}
Thus, genetic characteristics presents a powerful opportunity for a deeper understanding of speciation processes.


%% Models
%from Rangel et al 2018 (science, cradles/museums/graves):
%''spatially and temporally explicit, process-based models (18, 19),
%founded on a comprehensive suite of well-studied, widely accepted mechanisms,
%have the greatest potential to assess the complex and sometimes indeterminate
%interactions among underlying processes''
%and:
%"The second-most-influential model parameter was the maximum sustainable
%evolutionary rate realizable by a population (Hmax),
%which limits the adaptability of niche limits and evolutionary rescue (82–85)
%in the face of changing climates (figs. S16 to S19 and table S7).
%Low Hmax values indicate that niche traits have low genetic variance,
%low population growth rates, or both—preventing species from tracking
%and adapting to changing climates."
%==> what affects evolutionary rate?

However, the problem of investigating speciation processes is the time frame and study extent needed to go beyond mere correlation \cite{dormannCorrelationProcessSpecies}.
Because island biogeography processes and thus radiations generally take place at evolutionary time frames (millions of years),
comprehensive and meaningful data is difficult to obtain (\cite{didierLikelihoodTreeTopologies2017}, but see \cite{mitchellInferringDiversificationRate})
Therefore, there has been an interest in building process-based island models to test the causality of assumptions and hypotheses.
In a recent review, we indeed found a rapid increase in number of mechanistic simulation models for island biogeography \cite{leidingerBiodiversityDynamicsIslands2017}.
Although many of the reviewed models integrate multiple processes at a time, unfortunately,
most models failed to include micro-evolutionary or more explicit genetic processes %\cite{leidinger2017biodiversity}
(Leidinger \& Sarmento Cabral 2017).
One direction towards this is given by models using concepts such as ``protracted speciaton'' \cite{rosindellUnifiedModelSpecies2013}.
where speciation occurs with a delay from the actual
divergence event if the population remains isolated during that time.
But nature is usually more complex than this (cf.\ \cite{silvertownGhostCompetitionPhylogeny2004} but see \cite{herbenGhostHybridizationNiche2005}) 
and we expect genetic mechanisms to play a central role in speciation/radiation and species richness.
Questions, such as "What functional traits (e.g. relating to dispersal capacity,
reproduction, trophic ecology) are associated with high diversification rates within and across island systems?" %better another...
or "What is the influence of gene flow among islands and/or
between islands and mainland areas on speciation rates?", thus remain largely open \cite{patinoRoadmapIslandBiology2017}
and may require explicit consideration of evolutionary processes.
%% Yet, there are numerous examples for fascinating patterns of trait distributions and evolution on islands, such as island syndromes[, ...] or the fact of explosively radiating lineages (Hughes \& Eastwood 2006, Kocher 2004)
%%, while others remain one single species (Price/Chase) (eg. \cite{price2004speciation}).
%% Also: Ecological Opportunity and Adaptive Radiation, Stroud \& Losos 2016: ``Does the concept have predictive value[...]?'' \cite{stroud2016ecological}
%% Phylogenetic niche conservatism, phylogenetic signal and the relationship between phylogenetic relatedness and ecological similarity among species, Losos 2008: ecological vs. phylogenetic similarity \cite{losos2008phylogenetic}
Still, the most mechanisms integrated in island models [?] so far include only ecological
or geo-morphological processes (Rundell \& Price 2009, Esselstyn et al. 2011),
while genetic properties are hardly recognized in the island context.
Given the essential role islands have played in the development of evolutionary theories, it is surprising that these processes
are not implemented more often in island simulations.

%Another shortfall of the reviewed models was the lack of integrating ecological niches and adaption.

%Besides adaptive radiations(/evolution), changes in or different measures of geographic isolation (landscape complexity) can also give rise to non-adaptive radiations.

%% Aim, objectives:
%% In this study, we aim to address...
To assess the role of genetic processes, namely genetic linkage, for island radiations,
we present a mechanistic model that overcomes above-mentioned shortcomings like missing micro-evolutinary processes and lack of integrative design
by implementing genetic processes on a niche-explicit background
using an island biogeography simulation arena.
We will address the questions of what relative impact does (a) genetic linkage and
(b) reproductive isolation have on emerging species richness and composition,
and (c) which genetic and functional trait syndromes %% Which?
distinguish successful colonizer,
strong competitor and diversifying lineages.
Considering these phenomena, we expect speciation rates to be highest at intermediate levels of gene linkage --- low
levels leads to dissolving phenotype populations, whereas high levels hinder rapid adaption. %what Schiffers et al say about linkage and evolutionary rescue (basically the opposite of diversification?)


\section{Material and Methods}
%% Include simplified model description ``based on ODD''
%% scheduling (order of processes)
%% justify each decision with reference
%% Experimental design

\begin{figure}
  %\includegraphics{}
  \caption{Schematic of the model's individuals genetic architecture.
    Individuals are composed of a diploid set of linkage units which combine the genes coding for functional traits.}
  \label{schematic}
\end{figure}

We use an individual-based spatially and temporally explicit metacommunity model written in julia \cite{bezansonJuliaFreshApproach2017}.
The model considers competition between individuals, environmental niche adaptation, density independent mortality,
dispersal, sexual reproduction and mutation.
Individuals are characterized by a lineage identifier,
body mass, age and fitness.
Each individual's life history traits (table) are coded by mutable genes \cref{fig:schematic}.
The genes, in turn, may be combined to linkage units (``chromosomes'').
Genes in a linkage unit are considered a heredetary unit.
This means that they will always be inherited in junction.
Thus, only complete linkage units will be subject to recombination.

Building on the model of Cabral et al. (2017a, b), reproduction rates (as number of offspring),
growth rates and mortality probabilities follow the metabolic theory
of ecology (MTE, Brown et al.\ 2004). % instead of saying 'most', say which follow and which don't
In MTE, a biological rate $b$ depends on the temperature $T$ and individual mass $M$, scaling a base rate $b_0$ as:
\begin{equation}
  b=b_0*M^{3\over4}*e^{-{E_A}\over{k_B*T}}
  \label{eq:MTE}
\end{equation}
with $E_A$ the activation energy and $k_B$ the Boltzman constant.

Dispersal is controlled by a logistic dispersal kernel \cite{bullockjamesm.SynthesisEmpiricalPlant2016}
with parameters PAR1 and PAR2
for mean dispersal kernel and kernel shape (tail fatness), respectively.

A detailed protocol can be found in the appendix \cite{grimmStandardProtocolDescribing2006a} \cite{grimmODDProtocolReview2010}.



The simulation arena consists of a larger landmass (the mainland) holding an initial random species pool and a smaller,
initially unpopulated island some distance apart (user-defined in the experimental design).
% it would be nice to hav a table with all model configurations that can be defined by the user upon designing the experiment.
% This might help readers see that the model is general enough to be applied to any ecological arena (mainland-island being one).
During the course of the simulation, individuals may disperse from the mainland to the island, thus colonizing it. 
The complete state of the simulation is recorded every 1000 time steps for analysis.

\subsection{scenarios}
In the simulations, we modify the value and extent of
%(a)
gene linkage
%(b) reproductive isolation (or reproduction compatibility),
%(c) habitat diversity, and
%(d) topographical complexity
to asses their relative contribution on community composition.
Additionally, we simulate lineages in isolation with varying degrees of genetic linkage
and analyize emerging species numbers.
\cref{tab:scenarios} summarizes the parameters which were varied for the scenarios, their meaning and their values. %TODO: make table.
Each parameter combination was replicated %TODO: how much?
times.%, setting a different random seed for each replicate.

\subsection{Data recording}
We record individual data (properties and genome) at the start and end of a simulation run
and at intervals of 1000 time steps.% and each time the arena changes (geomorphological dynamics).
Additionally, we keep track of the genealogy of our individuals through time,
allowing us to validate inferred phylogenies.
Using the thus simulated data, we calculate phylogenies of the colonizing lineages, emerging species numbers and
trait space. % etc. [WHAT ELSE?]

\section{Results}

\subsection{Single species scenarios}
Species evolving on its own without interference from other species showed complex and dynamic speciation/differentiation patterns.
While low-linkage lineages adapted to the new environmental conditions on the island within some thousand timesteps to produce distinct populations,
high-linkage lineages were noticeably slower to differentiate.
Several 10.000 years later, however, high-linkage lineages began with strong radiations, thus surpassing species numbers as compared to low-linkage scenarios.

\begin{figure}
  %\includegraphics{}
  \caption{[2 x 2]
    Emerging numbers of species (and average population sizes) over time.
    Low-linkage lineages in COLOR1, high-linkage lineages in COLOR2.
    Top row: lineages simulated in isolation.
    Bottom row: simulated communities.}
  \label{specieslinkage}
\end{figure}

\begin{figure}
  %\includegraphics{}
  \caption{[2 x 2]
    Extinctions and numbers of species per lineage after XXXX timesteps against degree of genetic linkage.
    Top row: lineages simulated in isolation.
    Bottom row: simulated communities.}
  \label{specieslinkage}
\end{figure}

\begin{figure}
  %\includegraphics{}
  \caption{Trait space characteristics of lineages.
    (A) [Ordination] Trait space of successful colonizers compared to the mainland species pool. Each point represents a lineage.
    Arrows denote the traits.
    (B) [Ordination?] Linkage of traits of successful colonizers(/radiators?). Traits perpendicular to each other are inherited independently,
  same direction indicate tight linkage of traits.}
  \label{traitspace}
\end{figure}

\begin{figure}
  %\includegraphics{}
  \caption{[2 x 3]
    Exemplary genealogies of lineages in isolation for different degrees of genetic linkage.
    extant individuals.
    Columns: phylogeny and genealogy
    Rows: Lineage with low gene linkage,
    lineage with intermediate gene linkage,
    lineage with high gene linkage.}
  \label{phylogenies}
\end{figure}

\section{Discussion}
The fact that low-linkage (single) species speciated in general faster than their high-linkage counterparts
is in line with expectations given their higher potential for fast evolutionary response [as exemplified REFS].
What did come as a surprise, however was that, over time, high linkage lineages surpassed low linkage lineages
in the number of emerging species.
The reason for this is likely to be found in the high genetic variation throughout the high-linkage populations
due to genetic hitchhiking.
Combined with random drift, this genetic variation has a high potential of producing numerous species over time [non-adaptive radiation].
[EXAMPLES]
On the other hand, independent recombination of genes in low-linkage populations explains their quick adaptive response
to the conditions in their respective environments.
Once adapted and having reached a high degree of specialization, however, the genetic architecture leads to strong selection against
new and mainly deleterious mutations.
[EXAMPLES]

[SPECULATION:] In mixed community scenarios this effect becomes less clear as inter-lineage competition and red-queen-dynamics
present complex challenges.

%Assortative mating

%Clonality

Shortcomings. (/Processes/Factors not considered.)
(Dynamic) resource availability: assumed constant in model, but petentially more variation, complexity in community dynamics.
Interaction: in present model only resource competition, % I would call just say resource competition
but predation, parasitatism, herbivory, mutualism will increase complexity,
and should be rather aimed at future modle developments once mechanistic consequences of competition have been better investigated.
In particular, future model developments may concetrate on pollination,
which may have a high impact on evolution and establishment of both plant and animal species due to specialization. %examples for islands?
%[maybe some words on trophic interactions.]

In reality linkage between genes is not a binary decision, but rather a consequence of the distance between those genes.
The larger the distance, the higher the probability of crossing over during meiosis.
The size of the resulting genomic blocks or islands, however, may increase to several megabases in diverging population ---
a phenomenom coined as ``divergence hitchhiking'' (Via \& West 2008). % cite properly!
This mechanism is thought to facilitate sympatric speciation by reducing the detrimental effects of gene flow and
recombination between populations to speciation (Via 2011, but see Feder \& Nosil 2010).
%During incipient speciation, however, th

In this work we concentrated on genetic influences on speciation.
However, there is ongoing debate on how strongly genetic factors also impact extinction processes (Jamieson 2007).
For example, [discuss the evolutionary rescue studied by Schiffers et al. 2013]
Seeing that observed speciation rates are in fact net speciation rates highlights the need to mechanistically explore
these processes as well.
In fact, genetic processes might play an important role in the issue of extinction debt. %% cite Ludmilla, ask for refs

It should be noted that observed speciation and radiation events hinge upon arbitrary species concepts.
Increasingly more available population genome data make apparent that population genetic structure does not
always mirror classic systematics.
Therefore, we expect future studies to continue focussing on population level variablity when conducting genetic analyses.
\printbibliography

\end{document}
