\documentclass[a4paper]{scrartcl}
\usepackage[english]{babel}
\usepackage[
  backend=biber,
  style=authoryear
]{biblatex}
\addbibresource{refs.bib}

\usepackage[T1]{fontenc}
\usepackage[utf8]{inputenc}
\usepackage{csquotes}

\title{The relative roles of genetic, ecological and environmental factors in species radiations}
\author{Ludwig Leidinger}
\date{\today}
%
\begin{document}
\maketitle

\begin{abstract}
  Islands feature peculiar and fascinating biotas, often very distinct from any other island or continental area.
  [Examples of radiations/good and poor radiators, island syndromes etc.]
  Although island biogeography theory has seen an ever increasing development over the years,
  the mechanisms underlying these patterns are not fully understood/ not yet explained in a direct causal relationship.
  Specifically, some processes remain un- or under-explored, foremost of these adaption and micro-evolution.
  We used an individual-based mechanistic model implementing these processes and other commonly investigated ecological and
  environmental processes to assess the importance of drivers across different organisational and spatio-temporal levels
  on species trait distribution on islands.
  This enabled us to also take a closer look on the properties of lineages/species that actually managed to colonize islands,
  and which of them resulted in radiations[~].
  
  
\end{abstract}

\section{Introduction}
%% island biota have always fascinated and inspired researchers (Darwin, Wallace, ...)

%% This interest in islands and their recognition as highly suitable model systems resulted in the formulation of
%% important theoretical frameworks, foremost MacArthur and Wilson's seminal equilibrium theory of island biogeography (ETIB).
%% Later ... GDM.

%% These theories have in common a focus on species numbers and respective rates of migration, extinction and speciation.
%% While they are successful in describing species numbers, they are neutral in design (in the UNTB sense) and thus were not meant to consider species traits and compositions.

%% Yet, there are numerous examples for fascinating patterns of trait distributions and evolution on islands, such as island syndromes[, ...] or the fact that some lineages radiate explosively following
%% island colonisation, while others remain one single species (Price/Chase) (eg. \cite{price2004speciation}).

%% Also: Ecological Opportunity and Adaptive Radiation, Stroud \& Losos 2016: ``Does the concept have predictive value[...]?'' \cite{stroud2016ecological}
%% Phylogenetic niche conservatism, phylogenetic signal and the relationship between phylogenetic relatedness and ecological similarity among species, Losos 2008: ecological vs. phylogenetic similarity \cite{losos2008phylogenetic}

%% These patterns have been explained theoretically [here, here and here], but usually one at a time.
%% In the following, we aim to produce and thus explain these forementioned patterns using a single framework, and thus following the same set of (rather simple) rules.

%% Using an individual-based mechanistic model we investigated how genetic, ecological and environmental factors colonisation and radiation histories on islands,
%% and what trait combinations constitute successful colonizer and radiating species, respectively.

%% [NEW ANGLE:]
% Island biota have always fascinated and inspired researchers (Darwin, Wallace, ...). %TODO: refs
Islands are understood as suitable model systems for ecology and evolution due to their well definable characteristics like isolation. (e.g., Losos & Ricklefs 2009) %TODO: ref
Due to global change, however, island biota are also increasingly at risk.
While often featuring unique species communities and harboring countless endemics, %TODO: refs
islands are especially threatened by climate change, human impact and invasive species. %TODO: refs
This makes understanding the functioning and evolution of island communities a high priority objective.

Recently, a group of influential island biogeography researchers proposed a set of 50 questions in island biogeography that still need answering (Patiño et al. 2017).
A subset of these questions deals with the topic of speciation and radiations on islands.
Still, the proposed mechanisms so far include only ecological or geo-morphological processes,
while genetic properties are hardly recognized in the island context.
At the same time, an increasing number of studies investigates genetic processes and
their effect on reproductive/ecological isolation of populations (e.g., Via & West 2008, Via 2001 and references therein).

One of the first and to this day highly influential formal theories of island biodiversity is MacArthur and Wilson's seminal equilibrium theory of island biogeography (ETIB). %TODO: ref
In it, species richness on islands is the result of opposing rates of immigration and extinction, both of which in turn are modified by island size (area) and its distance from the mainland (isolation).
This relationship was since successfully applied to a multitude of islands %TODO: refs
, but also island like systems, such as coral reefs, lakes/ponds or forest fragments. %TODO: refs
This general applicability, however, cannot hide the fact that the ETIB fails to deliver a mechanistic explanation of the processes at play.
It has particularly been criticized for its lack of considering speciation. %TODO: refs

About a decade ago, island biogeography theory gained new momentum by Whittaker et al.'s General Dynamic Model (GDM). %TODO: ref
The conceptual framework traces island communities along the geomorphological trajectory of hotspot islands,
whith islands emerging from sea level, increasing over time in size and topographic complexity, before eroding and subsiding in the ocean again.
In doing so, the model incorporates concepts of niche theory, speciation and species succession.
Yet, there are still many processes and drivers deemed important, which are not considered by any formal theory.

%%TODO: here come a list of processes with examples
%% - genetic processes in general
%% - isolation processes

The problem of investigating these processes is the time frame needed for a proper evaluation.
Because islands biogeography takes into account evolutionary time frames (millions of years),
comprehensive and meaningful data is difficult to obtain.
In a recent review \cite{leidinger2017biodiversity}, we analyzed a powerful alternative to field experiments in island biogeography
--- mechanistic (/simulation) models.
Unfortunately, most island models failed to include more explicit genetic or micro evolution processes.
% reasons: underlying theories: Recently, a group of influential island biogeography researchers proposed a set of 50 questions in island biogeography that still need answering (Patiño et al. 2017).
%A subset of these questions deals with 
Instead, some models featured a concept known as ``protracted speciaton'' \cite{rosindellXXX}, where speciation occurs with a delay from the actual
divergence event if the population remains isolated during that time.
But nature is usually more complex than this and genetic mechanisms have important ramifications for speciation/radiation and species richness.
[EXAMPLES!]

For example, the rapid switching of pollinator syndromes in species of the genus \textit{Petunia} %use package for this!
seems to be a consequence of the tight genetic linkage between loci for pollination syndrome traits (Hermann et al. 2013) % cite properly!

Another shortfall of the reviewd models was the lack of integrating ecological niches and adaption.

Besides adaptive radiations(/evolution), changes in or different measures of geographic isolation (landscape complexity) can also give rise to non-adaptive radiations.

In the following, we present a model that remedies both of the adressed shortcomings by implementing genetic processes on a niche-explicit background
using an island biogeography framework.
We will adress the questions of what relative impact does (a) genetic linkage and (b) reproductive isolation have on emerging species numbers and diversity,
and how (c) landscape heterogeneity (habitat niche diversity) and (d) topographical complexity differently affect radiatons.
[further questions thus tackled: traits of colonizer species, ...]




\section{Material and Methods}
We use an individual based metacommunity model written in julia \cite{JULIALANG}.
Each individual is characterized by a set of genetically controlled traits and is submitted to processes of
establishment, competition, growth, (density independent) mortality, reproduction and dispersal.
Each of which process is affected by the individuals' trait characteristics.
The simulation arena consists of a larger landmass (the mainland) holding an initial pool of evolving populations and a smaller, initially unpopulated island some distance apart.
During the course of the simulation, individuals may disperse from the mainland to the island, thus colonizing it.
The complete state of the simulation is recorded every 1000 time steps for analysis.

A detailed protocol can be found in the appendix (\cite{grimm2010odd}, XXX).

\subsection{scenarios}
In the simulations, we modify the value and extent of
(a) gene linkage,
(b) reproductive isolation,
(c) habitat diversity, and
(d) topographical complexity
to asses their relative contribution on community composition.
\Cref{tab:scenarios} summarizes the parameters which were varied for the scenarios, their meaning and their values. %TODO: make table.
Each parameter combination replicated %TODO: how much?
times.

\subsection{Data recording}
We record the complete state of our simulation world at the end of a simulation run and each time the arena changes (geomorphological dynamics).
Additionally, we store the state of the simulation arena each 1000 %TODO: how much exactly?
time steps and record every individual that colonizes the island to track it in the phylogeny.

Using the thus simulated data, we calculate a phylogeny of the extant species, species numbers, etc. [WHAT ELSE?]

\section{Results}

\begin{figure}
  \includegraphics{}
  \caption{Number of linkage units and number of species per lineage XXX timesteps after colonization.
    [Histogram (with threshold category: strongly diversified lineages) or scatter plot.]}
  \label{specieslinkage}
\end{figure}

\begin{figure}
  \includegraphics{}
  \caption{Trait space characteristics of lineages.
    (A) [Ordination] Trait space of successful colonizers compared to the mainland species pool. Each point represents a lineage.
    Arrows denote the traits.
    (B) [Ordination?] Linkage of traits of successful colonizers(/radiators?). Traits perpendicular to each other are inherited independently,
  same direction indicate tight linkage of traits.}
  \label{traitspace}
\end{figure}


\section{Discussion}
Shortcomings. (/Processes/Factors not considered.)
(Dynamic) resource availability: assumed constant in model, but petentially more variation, complexity in community dynamics.
Interaction: in present model only indirect interaction via competition, but predation, parasitation, herbivory, mutualism will increase complexity.
Especially pollination is assumed to be an important interaction, because having a high impact on evolution and establishment of species due to specialization.
[maybe some words on trophic interactions.]

In reality linkage between genes is not a binary decision, but rather a consequence of the distance between those genes.
The larger the distance, the higher the probability of crossing over during meiosis.
The size of the resulting genomic blocks or islands, however, may increase to several megabases in diverging population ---
a phenomenom coined as ``divergence hitchhiking'' (Via \& West 2008). % cite properly!
This mechanism is thought to facilitate sympatric speciation by reducing the detrimental effects of gene flow and
recombination between populations to speciation (Via 2011, but see Feder \& Nosil 2010).
During incipient speciation, however, th


\printbibliography

\end{document}
